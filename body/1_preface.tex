\chapter{序論}
\section{背景}
近年,VRゲームは急速に発展しており,ゲーム体験を革新している.
その中で振動刺激は,視覚や聴覚の体験感を補助し,プレイヤーをより仮想空間に没入させる.仮想空間内での動作や事象と振動刺激を一致させることでさらにユーザーの没入感を高めることができる.

また,VRゲームは現実では実現不可能な魔法を疑似的に体験できる.
一般的なゲームコントローラーと違い,VRコントローラーを使い仮想空間内の魔法の杖を振ることで魔法を操る感覚を得ることでゲームへの没入感を高められる.

\section{課題}
剣で攻撃した時などの現実に存在している物体の振動刺激がフィードバックされるコンテンツは存在している.
しかし,空想上の存在である魔法の形状や出方に対してどのような振動を付随させればユーザーの没入感を高めることができるかを研究している例がない.
\section{目的}
そこで本研究では,前述した課題を解決するため,魔法の視覚エフェクトに対してどのような振動刺激を与えればユーザーの魔法体験感を向上させられるのかを調査する.

\section{本論文の構成}
本論文の構成について述べる.
1章では背景と課題,本研究の目的について述べた.
2章では本研究の関連技術,関連研究について述べる.
3章では本研究のシステムについて述べる.
4章では実験内容と評価方法について述べる.
5章では実験結果と考察について述べる.
6章では本研究の結論について述べる.
