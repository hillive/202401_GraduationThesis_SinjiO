\chapter{序論}
\section{背景}
近年VRゲームは急速に発展しており,ゲーム体験を革新している.
VRゲームは360度の視野や立体音響,VRヘッドセットによる身体の動きとの連動させることでプレイヤーの没入感をより高めている.
その中で振動刺激は,視覚や聴覚の体験感を補助し,プレイヤーをより仮想空間に没入させる役割を果たしている.
仮想空間内での動作や事象と振動刺激を一致させることでさらにユーザーの没入感を高めることができる.

また,VRゲームには魔法を疑似的に体験できるコンテンツが存在する.
VRコントローラ―を使用し,ボタンを押すことで魔法を放つものである.


\section{課題}
VRゲームでの魔法体験コンテンツの課題として,魔法の形状ごとにどのような振動フィードバックを与えればプレイヤーの魔法体験感を向上させられるのか分かっていないことが挙げられる.
VRゲーム上での剣やヤリ等の物理攻撃の振動フィードバックについて,日常生活での経験上どのような振動フィードバックがあるのか具体的に表現できる.
しかし,魔法は現実に存在しないため,魔法を放ったときにどのような振動フィードバックがあるのかよくわかっていない.

% 剣等の攻撃は日常生活の経験上物理的な振動フィードバックは想像の範囲内である.
% しかし,空想上の存在である魔法の振動フィードバックがどのようなものかは分かっていない.

\section{目的}
そこで本研究では,前述した課題を解決するためにVR上で放った魔法の視覚エフェクトと数種類の振動刺激を組み合わせてユーザーに呈示するシステムを作成する.
そのシステムを使用し,魔法の視覚エフェクトに対してどのような振動刺激を与えればユーザーの魔法体験感を向上させられるのかを調査することを目的とする.

\section{本論文の構成}
本論文の構成について述べる.
1章では背景と課題,本研究の目的について述べた.
2章では本研究の関連技術,関連研究について述べる.
3章では本研究のシステムについて述べる.
4章では本システムの実装について述べる.
5章では実験結果と考察について述べる.
6章では本研究の結論について述べる.
