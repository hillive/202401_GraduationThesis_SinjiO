\chapter{序論}
\section{背景}
魔法使いという存在はマンガ、アニメ、ゲーム等のさまざまなコンテンツに度々登場している.

近年VRゲームが登場し,魔法を疑似的に体験できる.
魔法にはさまざまな
魔法を使用する際、一定のモーションをすることで魔法を放っているシーンをよく見かける.そのため,魔法を使用するときは一定のモーションをするというイメージがついている.

しかし、ゲームで魔法を使うときは、ボタンを押すのみでモーションをすることがないのでユーザーの没入感が高まらない。



%%%%%%%%%%%%%%%%%%%%%%%%%%%%%%%%%%%%%%%%%%%%%%%%%%%%%%%%%%%%%%%%%%%%%%%
\begin{comment}
    \textblockcolour{pink}
    \begin{textblock}{4}(16, 1)
        \noindent
        【5】目次に続いて本文
    \end{textblock}
    
    \textblockcolour{PowderBlue}
    \begin{textblock}{12}(6, 6)
    \noindent
    以降、1章 (chapter)、1.1節 (section)、1.1.1項 (subsection) と呼ぶ
    \end{textblock}
    
    \begin{textblock}{6}(8, 12)
    1章の最初の節が1.1
    
    1.1節の最初の項が1.1.1
    \end{textblock}
    
    
    \begin{textblock}{2}(0.5, 13.5)
    \noindent
    段落1行目を字下げをする
    \end{textblock}
    
    \begin{textblock}{2}(0.5, 18)
    \noindent
    内容の区切りに合わせて段落を分ける
    \end{textblock}
    
    \begin{textblock}{6}(11, 26.5)
        本文の開始ページを1とする
    \end{textblock}
\end{comment}
%%%%%%%%%%%%%%%%%%%%%%%%%%%%%%%%%%%%%%%%%%%%%%%%%%%%%%%%%%%%%%%%%%%%%%%

\section{課題}
振動は生活の身近にあるもので、食材を切るとき、走行中の車に乗っているときなど様々な場面で感じるものだ.それゆえに魔法に関しても同様に振動が発生していると考えた.


また,魔法には
しかし,魔法の形状や出方に対してどのような振動を付随させればユーザーの没入感を高めることができるかを研究している例がない。



\section{目的}
そこで本研究では、VR空間内での魔法の形状に合わせて魔法の杖を振動させることでユーザーの没入感を高めるシステムを提案する。

\section{本論文の構成}
本論文の構成について述べる.
1章では背景と課題,本研究の目的について述べた.
2章では本研究の関連技術,関連研究について述べる.
3章では本研究のシステムについて述べる.
4章では実験内容と評価方法について述べる.
5章では実験結果と考察について述べる.
6章では本研究の結論について述べる.




%%%%%%%%%%%%%%%%%%%%%%%%%%%%%%%%%%%%%%%%%%%%%%%%%%%%%%%%%%%%%%%%%%%%%%%
\begin{comment}
    \textblockcolour{PowderBlue}
    \begin{textblock}{10}(6.5, 15.8)
        見出しの深さの最大値は研究室や分野によって異なる。
        
        教員の指示に従うこと。一般論として4段は深すぎ?
    \end{textblock}
\end{comment}
%%%%%%%%%%%%%%%%%%%%%%%%%%%%%%%%%%%%%%%%%%%%%%%%%%%%%%%%%%%%%%%%%%%%%%%
