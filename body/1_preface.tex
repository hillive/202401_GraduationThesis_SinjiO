\chapter{序論}
\section{背景}
近年VRゲームは急速に発展しており,ゲーム体験を革新している.
VRゲームは360度の視野や立体音響,VRヘッドセットによる身体の動きとの連動近年,VRゲーなどによりプレイヤーの没入感を高めている.
その中で振動刺激は,視覚や聴覚の体験感を補助し,プレイヤーをより仮想空間に没入させる役割を果たしている.
仮想空間内での動作や事象と振動刺激を一致させることでさらにユーザーの没入感を高めることができる.

また,VRゲームには魔法を疑似的に体験できるコンテンツが存在する.
VRコントローラ―を使用し,ボタンを押すことで魔法を放つものである.


\section{課題}
VRゲームでの魔法体験コンテンツの課題として,魔法の形状ごとにどのような振動フィードバックを与えればプレイヤーの魔法体験感を向上させられるのか分かっていないことが挙げられる.
VRゲーム上での剣やヤリ等の物理攻撃について,日常生活での経験上どのような振動フィードバックがあるのか具体的に表現できる.
しかし,魔法は現実に存在しないため,魔法を放ったときにどのような振動フィードバックがあるのかよくわかっていない.

% 剣等の攻撃は日常生活の経験上物理的な振動フィードバックは想像の範囲内である.
% しかし,空想上の存在である魔法の振動フィードバックがどのようなものかは分かっていない.

\section{目的}
そこで本研究では,前述した課題を解決するため魔法の視覚エフェクトに対してどのような振動刺激を与えればユーザーの魔法体験感を向上させられるのかを調査する.
ただ単に一定の振動刺激を与えるのではなく,魔法視覚エフェクトの時間推移に伴い振動の強弱を調整することで,よりユーザーの魔法体験を向上させられると考えた.

\section{本論文の構成}
本論文の構成について述べる.
1章では背景と課題,本研究の目的について述べた.
2章では本研究の関連技術,関連研究について述べる.
3章では本研究のシステムについて述べる.
4章では本システムの実装について述べる.
5章では実験結果と考察について述べる.
6章では本研究の結論について述べる.
