\chapter{実装}
\section{システム構成}

本システムの構成を\figref{sys}に示す.本システムは主にHMD,トラッカー,PC,角材,振動モーターで構成される.
被験者にHMDを装着する。HMDはPCから送信された映像を描画する.

\begin{figure}[b]
\centering
\includegraphics[clip,width=10cm]{./fig/systemP.png}
\caption{システム構成}\label{sys}
\end{figure}

被験者はHMDを装着した状態でコントローラーを持つ.コントローラーはトラッカーにより座標と角度を取得することで魔法の杖をVRに反映させている.

コントローラーを\figref{controller}に示す。
\begin{figure}[b]
\centering
\includegraphics[clip,width=10cm]{./fig/controller.png}
\caption{コントローラ―}\label{controller}
\end{figure}







%%%%%%%%%%%%%%%%%%%%%%%%%%%%%%%%%%%%%%%%%%%%%%%%%%%%%%%%%%%%%%%%%%%%%%%
\begin{comment}
    \begin{textblock}{2}(1, 16.5)
        空行→
    \end{textblock}
    
    \begin{textblock}{2}(1, 18.5)
        字下げ→
    \end{textblock}
        
    \begin{textblock}{2}(1, 20.5)
        空行→
    \end{textblock}
    
    \begin{textblock}{11}(9, 20.5)
        ←読点までが元の文なので文献番号はその後につける
    \end{textblock}
    
    \begin{textblock}{7}(14, 26.5)
        ↑同じく読点までが元の文なので
    
        "」"と文献番号はその後につける
    \end{textblock}
\end{comment}
%%%%%%%%%%%%%%%%%%%%%%%%%%%%%%%%%%%%%%%%%%%%%%%%%%%%%%%%%%%%%%%%%%%%%%%    




%%%%%%%%%%%%%%%%%%%%%%%%%%%%%%%%%%%%%%%%%%%%%%%%%%%%%%%%%%%%%%%%%%%%%%%
\begin{comment}
    \begin{textblock}{11}(9, 7.5)
        \noindent
        ↑間接引用では節末・文末の句読点の「前」に文献番号をつける
    \end{textblock}
\end{comment}
%%%%%%%%%%%%%%%%%%%%%%%%%%%%%%%%%%%%%%%%%%%%%%%%%%%%%%%%%%%%%%%%%%%%%%%




