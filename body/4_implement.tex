\chapter{実験}

評価実験では,VR上に魔法の杖を表示しそこから数種類の形状の魔法を放ち、それに伴い振動モーターを振動させ,魔法の形状に対して振動パターンがユーザーの没入感にどのような影響を与えるのかを調査する.

被験者として--に評価してもらった.

\section{実験手順}
本実験の手順を以下に示す.





%%%%%%%%%%%%%%%%%%%%%%%%%%%%%%%%%%%%%%%%%%%%%%%%%%%%%%%%%%%%%%%%%%%%%%%
\begin{comment}
    \begin{textblock}{2}(1, 16.5)
        空行→
    \end{textblock}
    
    \begin{textblock}{2}(1, 18.5)
        字下げ→
    \end{textblock}
        
    \begin{textblock}{2}(1, 20.5)
        空行→
    \end{textblock}
    
    \begin{textblock}{11}(9, 20.5)
        ←読点までが元の文なので文献番号はその後につける
    \end{textblock}
    
    \begin{textblock}{7}(14, 26.5)
        ↑同じく読点までが元の文なので
    
        "」"と文献番号はその後につける
    \end{textblock}
\end{comment}
%%%%%%%%%%%%%%%%%%%%%%%%%%%%%%%%%%%%%%%%%%%%%%%%%%%%%%%%%%%%%%%%%%%%%%%    




%%%%%%%%%%%%%%%%%%%%%%%%%%%%%%%%%%%%%%%%%%%%%%%%%%%%%%%%%%%%%%%%%%%%%%%
\begin{comment}
    \begin{textblock}{11}(9, 7.5)
        \noindent
        ↑間接引用では節末・文末の句読点の「前」に文献番号をつける
    \end{textblock}
\end{comment}
%%%%%%%%%%%%%%%%%%%%%%%%%%%%%%%%%%%%%%%%%%%%%%%%%%%%%%%%%%%%%%%%%%%%%%%




