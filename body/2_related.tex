\chapter{関連研究}

%%%%%%%%%%%%%%%%%%%%%%%%%%%%%%%%%%%%%%%%%%%%%%%%%%%%%%%%%%%%%%%%%%%%%%%
\begin{comment}
\begin{textblock}{6}(14.5, 22)
  ←図のキャプションは図の下
\end{textblock}
\end{comment}
%%%%%%%%%%%%%%%%%%%%%%%%%%%%%%%%%%%%%%%%%%%%%%%%%%%%%%%%%%%%%%%%%%%%%%%

% 


\section{スマートフォンにおける多様な振動フィードバックが被験者の印象に与える影響}
スマートフォンにおける多様な振動フィードバックが被験者の印象に与える影響{参考入れる}の研究を白神らが行った.この研究では振動パターンがユーザーに与える印象に焦点を当て,約250通りの振動パターンからユーザーがどのような印象を持ったのかを調査したものである。

% \figref{vive}
は振動パターンの一例です。これはジョジョに振動強度が上昇するもので,この振動刺激はユーザーに力強いという印象を与えます。



しかし,この研究は,スマホという掌の上での振動でしかなく,さらに大きな物体での振動についての調査は行っていない.
また,実際に存在しているものに関する振動についての調査なので,魔法という非現実的なものに対する振動については明かされていない.





%%%%%%%%%%%%%%%%%%%%%%%%%%%%%%%%%%%%%%%%%%%%%%%%%%%%%%%%%%%%%%%%%%%%%%%
\begin{comment}
  \begin{textblock}{6.5}(1, 18)
    \noindent
    【16,18】図番号は章ごとの通し番号で抜けがない
  \end{textblock}
  
  \begin{textblock}{7}(13, 22)
    ←本文で説明がない図は載せない
  \end{textblock}
\end{comment}
%%%%%%%%%%%%%%%%%%%%%%%%%%%%%%%%%%%%%%%%%%%%%%%%%%%%%%%%%%%%%%%%%%%%%%%