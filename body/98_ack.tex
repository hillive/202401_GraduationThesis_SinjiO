% 謝辞
\theacknowledgments

%%%%%%%%%%%%%%%%%%%%%%%%%%%%%%%%%%%%%%%%%%%%%%%%%%%%%%%%%%%%%%%%%%%%%%%

\begin{comment}
    \textblockcolour{lime}
    \begin{textblock}{12}(6, 6)
        謝辞はなくてもよい。
        
        謝辞・参考文献・ソースコードなどの付録は本文に含まれない。
    \end{textblock}
    
    \begin{textblock}{12}(6, 12)
        論文全体で句読点の形式を統一する。
    
        このサンプルでは謝辞の句読点が他の章のそれと異なっている。
    
        どの句読点形式で執筆をするか指導教員に確認すること。
    \end{textblock}
\end{comment}
%%%%%%%%%%%%%%%%%%%%%%%%%%%%%%%%%%%%%%%%%%%%%%%%%%%%%%%%%%%%%%%%%%%%%%%

本研究を進めるにあたり 井上 教授准教授, 様をはじめとした皆様からご指導,ご意見を頂きましたことに心より御礼申し上げます.

そして,レジュメや卒業論文を書く際にご指導頂きました島谷先輩,荒川先輩に御礼申し上げます.

最後になりましたが,日常の議論を通じて多くの知識や示唆を頂いた井上研究室の皆様に御礼申し上げます.ありがとうございました.
