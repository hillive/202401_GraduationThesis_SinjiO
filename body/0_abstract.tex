近年,VRゲームは急速に発展しており,ゲーム体験を革新している.
その中で振動刺激は,視覚や聴覚の体験感を補助し,プレイヤーをより仮想空間に没入させる.仮想空間内での動作や事象と振動刺激を一致させることでさらにユーザーの没入感を高めることができる.

これまで剣で攻撃した時などの現実に存在している物体の振動刺激は研究されてきた.しかし,空想上の存在である魔法の形状や出方に対してどのような振動を付随させればユーザーの没入感を高めることができるかを研究している例がない。
そこで本研究では、VRを用いてユーザーの魔法体験における没入感を向上させるシステムを提案、開発し、視覚エフェクトと振動刺激の組み合わせによってユーザーの没入感にどのような影響を与えるのかを調査する.システム内で魔法を放ったと同時に,その魔法に適した振動刺激を与えるシステムを作成する.

評価実験では3種類の視覚エフェクトと4種類の振動パターンを用意し,全12通りの組み合わせを体験してもらいそれぞれどの程度一致していると感じたか5段階で評価してもらった.結果として,魔法の形状ごとに振動パターンを変化させることでプレイヤーの魔法体験感をより高めることができると結論付けた.

今後の展望として,振動のタイミングの改善,モーションを用いた魔法エフェクトの表示,別属性の魔法や移動魔法など攻撃魔法以外の振動刺激の調査などが挙げられた.