近年VRゲームは急速に発展しており,ゲーム体験を革新している.
VRゲームは360度の視野や立体音響,VRヘッドセットによる身体の動きとの連動させることでプレイヤーの没入感を高めている.
その中で振動刺激は,視覚や聴覚の体験感を補助し,プレイヤーをより仮想世界に没入させる役割を果たしている.
また,VRゲームには魔法を疑似的に体験できるコンテンツが存在する.

剣やヤリ等の物理攻撃について,日常生活での経験上どのような振動フィードバックがあるのか具体的に表現できる.
しかし,VRゲームでの魔法体験コンテンツの課題として,魔法の形状ごとにどのような振動フィードバックを与えればプレイヤーの魔法体験感を向上させられるのか分かっていない.

そこで本研究では,前述した課題を解決するためにシステム内で魔法を放ったと同時に,その魔法に適した振動刺激を与えるシステムを作成する.
そのシステムを使用し,魔法の視覚エフェクトに対してどのような振動刺激を与えればユーザーの魔法体験感を向上させられるのかを調査することを目的とする.

評価実験では 3 種類の視覚エフェクトと 4 種類の振動パターンを用意し, 全 12 通りの組み合わせを体験してもらいそれぞれどの程度一致していると感じたか 5 段階
で評価してもらった. 
結果として, 魔法の形状ごとに振動パターンを変化させることでプレイヤーの魔法体験感をより高められることがわかった.
今後の展望として, 振動のタイミングの改善, モーションを用いた魔法エフェクトの表示, 別属性の魔法や移動魔法など攻撃魔法以外の振動刺激の調査などが挙げられた.
