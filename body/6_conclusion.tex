\chapter{結論}
本研究では,VRを用いてユーザーの魔法体験における没入感を向上させるシステムを提案・開発し,視覚エフェクトと振動刺激の組み合わせによってユーザーの没入感にどのような影響を与えるのかを調査した.

評価実験では,開発したシステムで被験者に3種類の魔法の視覚エフェクトと4種類の振動刺激の組み合わせを体験してもらった.
その結果,事前アンケートで得られた振動強度の時間推移と同じような形状の振動パターンが高い評価になった.
これにより,魔法の形状ごとに振動パターンを変化させることでプレイヤーの魔法体験感をより高められることが分かった.

本実験の問題点として,振動パターンが少なかったことを挙げた.
4種の振動パターンのそれぞれに波形は同じだがタイミングを少しずらした振動刺激などを用意し,より細かく分析する必要があった.

今後の展望として,前述した問題点の改善に加え,振動のタイミングの改善やモーションを用いた魔法エフェクトの表示や別属性の魔法や移動魔法など,攻撃魔法以外の振動刺激の調査が挙げられる.
  